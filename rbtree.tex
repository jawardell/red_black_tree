\documentclass[notitlepage]{report}
\usepackage[left=1in, right=1in, top=1in, bottom=1in]{geometry}

\usepackage{titling}
\usepackage{lipsum}

\pretitle{\begin{center}\Huge\bfseries}
\posttitle{\par\end{center}\vskip 0.5em}
\preauthor{\begin{center}\Large\ttfamily}
\postauthor{\end{center}}
\predate{\par\large\centering}
\postdate{\par}

\title{An Exploration of RB Trees and Applications}
\author{KD Adkins, Beth Norton, Omar Batyah, Austin Hufstetler, Joanne Wardell}
\date{\today}
\begin{document}

\maketitle
\thispagestyle{empty}

\begin{abstract}
	A red black tree is a self balancing binary tree. The nodes on a red black tree
	have an extra attribute which signifies their color. There are two colors which are
	used in Red Black trees. The color attriubte per node is used as a tool for completing 
	an approximate balancing of the tree. Some properties of RB trees are as follows: 
	\begin{itemize}
		\item Every node is either red or black.
		\item The root is always black.
		\item All terminal leaves are black.
		\item The children of red nodes are black.
		\item Any path from a given node to a leaf contains the same number of black nodes.
	\end{itemize}
	Our research included the study of Red Black trees and thier applications
	in specific systems. One system in which we used 
	a red black tree was the ability to locate a phone number and caller in a database. 
	During this project, we explored the implementation of RB trees 
	and pondered some of the draw backs of using them. We used the data structure 
	in various modeled systems and propsed ways to improve the implementation of 
	the RB trees per system. 
\end{abstract}
\end{document}
